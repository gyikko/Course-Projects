\hypertarget{index_intro_sec}{}\section{Introduction}\label{index_intro_sec}
The transit simulator and visualization project simulates buses\textquotesingle{} information, stops\textquotesingle{} locations, the amount of loaded passengers, and the number of unloaded passengers, when they are running on provided routes, and visualized the information by using red dots to present buses, and showing the movement on a real map.\hypertarget{index_obtain_sec}{}\subsection{How to obtain?}\label{index_obtain_sec}
To obtain the code, simpily download the project from github. Or, you can choose to use git clone command to copy the project in your local devices.\hypertarget{index_config_sec}{}\subsection{How to configure?}\label{index_config_sec}
To run the program, first you need to make sure you\textquotesingle{}re running the code in Unix environment. Meanwhile, another tool you need for running the code is a browser.\hypertarget{index_compile_sec}{}\subsection{How to compile?}\label{index_compile_sec}
After downloading the code, use {\ttfamily cd} command to change the current location to src file under project directory. Then, use {\ttfamily make} command to compile the code, all the files will be compiled directly. You can also use {\ttfamily make clean} command to clean all the built objects.\hypertarget{index_execute_sec}{}\subsection{How to execute?}\label{index_execute_sec}
When you compile the program, you will be under src directory. However, to execute the program, yoe need to use {\ttfamily cd} command again to change to previous directory, project. Under project deirectory, use {\ttfamily ./build/bin/vis\+\_\+sim $<$port\+\_\+number$>$} to ececute. Then, to see the simulator, open a web browser, and enter the address in your address bar, {\ttfamily \href{http://127.0.0.1:}{\tt http\+://127.\+0.\+0.\+1\+:}$<$port\+\_\+number$>$/web\+\_\+graphics/project.html}.\hypertarget{index_discussion_sec}{}\subsection{How Bus Factory Can be Implemented?}\label{index_discussion_sec}
\hyperlink{classBus}{Bus} Factory can be implemented into two ways, abstract factory method and concrete bus factory. Though concrete bus factory is implemented, I will discuss both methods in here.
\begin{DoxyItemize}
\item What are these two methods? Abstract and concrete factory patterens are both able to do the same thingds, creating products. However, abstract factory pattern creates famlilies instead of creating products like concrete factory pattern. Here are two U\+ML diagrams to show the structures of these two factory patterns. Abstract \hyperlink{classBus}{Bus} Factory  Concrete \hyperlink{classBus}{Bus} Factory 
\item Differences Between the Two Factory Patterns
\end{DoxyItemize}
\begin{DoxyEnumerate}
\item The first difference is what mentioned above, abstract factory pattern mainly create sub-\/families which are responsible to create dependent productswhile concrete factory pattern creates products.
\item It is easy to understand the concrete factory pattern, it is a simple inheritance to finish produce the products. However, abstract factory pattern composites.
\end{DoxyEnumerate}
\begin{DoxyItemize}
\item Pros Two Factory Patterns The most distinguised benefits of useing a concrete factory is the simplicity. Because one only needs to use on factory to produce multiple types of products, the code designing phrase can save lots of time and efforts. However, based on the structure of the abstract factory method, the products produced by each sub-\/families are more compatible with each other. Meanwhile, because the factories are saperated into multiple types, product creation codes can be extracted as well, so it is easier to expand or support only one part of the code. New variants of products can be added easily.
\item Cons of Two Factory Patterns The cons of the two factories are each other\textquotesingle{}s pros. Though it seems like abstract factory pattern have many benefits, it is complicated to build a abstract factory, which can be counted as con. However, the convenience brought many cons when implementing concrete factory pattern. When you want to expand functionalities of a single product, it is difficult to add into the code. Products and products are less compatible.
\item Pattern Chosen for the Project This project used concrete factory pattern. Because for this project, so far I do not need to fulfill more functionality, in order to be simple, I chose to implement concrete factory pattern. 
\end{DoxyItemize}